In this chapter, we explore the detail for posterior inference and give specification to how we . In section \ref{ch4:1}, we give definition to the variational inference method that we use to posterior approximation. Then a detailed derivation for the evidence lower bound(ELBO) of our proposed model will be given in section\ref{ch4:2}. Finally, section \ref{ch4:3} will demonstrate the algorithm that to optimization process of the LKJTM.
\section{Posterior Inference}\label{ch4:1}
Since the exact inference of the posterior is intractable in real application, we employed approximation scheme for the posterior inference. The popular approaches are Markov Chain Monte Carlo Method (MCMC) and Variational Inference(VI)\cite{blei_variational_2006,hoffman_stochastic_2013}. Gibbs sampling is one of the MCMC method and it is fast to compute the approximation and easy to the implementation. Then, Variational EM algorithm to be carried out for maximizing the likelihood over all word in corpus in the document. An alternative way to perform estimation is Monte Carlo method.
\subsection{Variational Inference}
Given that posterior approximiation is not alway practical in real world application. Approximation method are necessary to be apply. Their are two main approach for the posterior approximation: Markov Chain Monte Carlo(MCMC) Variational Inference. Variational Inference is a method approximating the posterior in optimization fashion.
\subsection{Stochastic Variational Inference}
Stochastic Variational Inference (SVI)\cite{hoffman_stochastic_2013} is a scalable variant of variational inference, which enables mini-batching to split dataset and train for each epochs, then become a standard of optimization for probabilistic models.
Two main improvement are made by the SVI: stochastic optimization and noisy gradient.
% TODO Collapsing parameters
\subsection{Collapsing Parameters}
In original LDA model, the parameter z is responsible for sampling topic assignment for each word position in every single document. Collapsing parameters\cite{srivastava_autoencoding_2017} introduced to reduce the latent variable $ z $ in the generative process in hence to speed up computation.
\begin{align}\label{eq:cp}
w_d\sim\prod_{n=1}^{N_d}\text{Cat}(\sigma(\beta_{w_{dn}}\theta_d))
\end{align}
The trick in equation \ref{eq:cp} rewrite the original LDA word drawing process, and hence define a new evidence lower bound for the topic model.
% TODO Blei's notes
\subsection{Autoencoding Variational Bayes (AEVB)}
Generally when we optimize a variational parameter, it is neceasary to derive a ELBO and then derive the optimization step for gradient descent. While amortized inference latent variable z is parameterized by two inference network  $ \mu_{\phi(x),\sigma_{\phi}(x)} $ .
% TODO UCLA notes
\begin{align}
z = \mathcal{N}(\mu_{\phi}(x_i), \sigma_{\phi}(x_i))
\end{align}
\begin{align}
\mathcal{L}=\mathbb{E}_{z\sim\mathcal{N}(\mu_{\phi}(x_i),\sigma_\phi(x_i))}\left[\log p_\theta(x_i|z)\right]-D_{KL}\left(q_\phi(z|x_i)||p(z)\right)
\end{align}
\subsection{Reparameterization trick}
The drawback of amortized inference is that, sampling from normal distribution parameterizing $ \mu_{\phi(x),\sigma_{\phi}(x)} $ could lead to high variance outcome and hamper the inference performance. For the reason, taking reparamterization trick\cite{kingma_auto-encoding_2014} to transform as equation \ref{eq:rt},
\begin{align}\label{eq:rt}
z= \mu_{\phi}(x_i)+\epsilon\sigma_\phi(x_i)\text{, }\epsilon\sim\mathcal{N}(0,1)
\end{align}
where $ \epsilon $ is a sample from normal distribution $ \mathcal{N}(0,1) $. and so the modified ELBO becomes equation \ref{eq:elbo_rt},
\begin{align}\label{eq:elbo_rt}
\mathcal{L}=\mathbb{E}_{\epsilon\sim N(0,1)}\left[\log p_\theta(x_i|\mu_{\phi}(x_i)+\epsilon\sigma_\phi(x_i))\right]-D_{KL}\left(q_\phi(z|x_i)||p(z)\right)
\end{align},
% Inference Equation
% Variational Inference
% KL-divergence for the logistic normal distribution
\section{Evidence Lower Bound(ELBO)}\label{ch4:2} To perform Variational inference, it is essential to derive the Evidence Lower Bound (ELBO) first as the objective function for the optimization. By 
\begin{align}\label{eq:elbo_1}
\mathcal{L}\geq&\mathbb{E}_q[\log p(W,Z,\theta,\Sigma)]-\mathbb{E}_q[\log q(Z,\theta,\Sigma)]\\
=&\sum_{d=1}^{D}\sum_{n=1}^{V}\mathbb{E}_q[\log p(w_{d,n}|z_{d,n},\beta)]+\sum_{d=1}^{D}\sum_{n=1}^{V}\mathbb{E}_q[\log p(z_{d,n}|\theta_d)]\\
&+\sum_{d=1}^{D}\mathbb{E}[\log p(\theta_d|\mu,\Sigma)]+\mathbb{E}_q[\log p(\Sigma|\gamma)]-\sum_{d=1}^{D}\sum_{n=1}^{V}\mathbb{E}_q[\log q(z_{d,n}|\alpha_{d,n})]\\
&-\sum_{d=1}^{D}\mathbb{E}_q[\log q(\theta_d|\lambda_d,\nu_d)]-\mathbb{E}_q[\log q(\Sigma|\phi)]
\end{align}
% TODO Monte Carlo Estimate
The expectation log likelihood term in REPLACE can be efficiently appriximated by the Monte Carlo sampling method,
\begin{align}\label{eq:elbo_2}
\mathcal{L}\approx\frac{1}{S}\sum_{s=1}^{S}p(W|\theta^{(s)})
\end{align}
\paragraph{Collapsing Parameters}
\begin{align}\label{eq:elbo_3}
\mathcal{L}&\geq\sum_{d=1}^{D}\int\int q(\theta_d)\log\frac{p(W_d|\theta_d,\beta)p(\theta_d|\mu,\Sigma)p(\Sigma|\gamma)}{q(\theta_d)q(\Sigma)}d\theta_d d\Sigma\\
&=\sum_{d=1}^{D}\left(\mathbb{E}_{q(\theta_d)}\left[\log p(W_d|\theta_d,\beta)\right]-KL(q(\theta_d)||p(\theta_d|\mu,\Sigma))\right)-KL(q(\Sigma)||p(\Sigma|\gamma))
\end{align}
Here we define amortized inference, a optimization technique which perform inference by defined neural networks. $ \mu_\theta(w) $ and $ \sigma_\theta(w) $ are two inference networks take input from the word $ w $. Then a output is generated by the Normal distribution parameterized by $ \mu_\theta(w) $ and $ \sigma_\theta(w) $.
\begin{align}\label{eq:elbo_4}
=&\sum_{d=1}^{D}\left(\mathbb{E}_{q(\theta_d)}\left[\log p(w_d|\mathcal{LN}(\mu_{\theta_d}(w_d),\sigma_{\theta_d}(w_d)),\beta)\right]-KL(q(\theta_d)||p(\theta_d|\mu,\Sigma))\right)\\
&-KL(q(\Sigma)||p(\Sigma|\gamma))
\end{align}
To apply reparameterization trick, we take transformation from normal distribution to $ \theta=\mu+\epsilon\sigma^{1/2} $ where $ \epsilon\sim N(0,1) $, as  equation \ref{eq:elbo_5}
\begin{align}\label{eq:elbo_5}
=&\sum_{d=1}^{D}\left(\mathbb{E}_{q(\epsilon)}\left[\log p(w_d|\sigma(\mu_{\theta_d}(w_d)+\epsilon\sigma_{\theta_d}(w_d)),\beta)\right]-KL(q(\theta_d)||p(\theta_d|\mu,\Sigma))\right)\\
&-KL(q(\Sigma)||p(\Sigma|\gamma))
\end{align}
we also apply the minibatch to make able the model perform by subsampling the document collection. By equation \ref{eq:elbo_6}
\begin{align}\label{eq:elbo_6}
\tilde{\mathcal{L}}=&\frac{\mathcal{D}}{|\mathcal{B}|}\sum_{d\in\mathcal{D_B}}\left(\mathbb{E}_{q(\epsilon)}\left[\log p(w_d|\sigma(\mu_{\theta_d}(w_d)+\epsilon\sigma_{\theta_d}(w_d)),\beta)\right]-KL(q(\theta_d)||p(\theta_d|\mu,\Sigma))\right)\\
&-KL(q(\Sigma)||p(\Sigma|\gamma))
\end{align}
The KL-divergence for the logistic-normal distribution is given as equation \ref{eq:elbo_7} closed-form expression,
\begin{align}\label{eq:elbo_7}
\text{KL}(q(\theta_d)||p(\theta_d|\mu,\Sigma))=-\frac{1}{2}\left(tr(\sigma_1^{-1}\sigma_0)+(\mu_1-\mu_0)\Sigma_1^{-1}(\mu_1-\mu_0)-K+\log\frac{|\sigma_1|}{|\sigma_0|}\right)
\end{align}
so the ELBO then becomes \ref{eq:elbo_8}
\begin{align}\label{eq:elbo_8}
\tilde{\mathcal{L}}=&\sum_{d=1}^{D}\left[-\frac{1}{2}\left(tr(\sigma_1^{-1}\sigma_0)+(\mu_1-\mu_0)\Sigma_1^{-1}(\mu_1-\mu_0)-K+\log\frac{|\sigma_1|}{|\sigma_0|}\right)\right]\\
&+\mathbb{E}_{\epsilon\sim\mathcal{N}(0,I)}\left[w_d^\top\log\sigma(\beta(\mu_0+\sigma_0^{1/2}\epsilon))\right]-KL(q(\Sigma)||p(\Sigma|\gamma))
\end{align}
% TODO Transformer Loss
\section{Transformer Loss}
It is suggested that using cross entropy loss for transformer training. In equation \ref{eq:crossentropy}, . It is worth mention that, the 
\begin{align}\label{eq:crossentropy}
L_{\text{CrossEntropy}}=-\frac{1}{V}\sum_{i=1}^{V}y_i\cdot\log(\hat{y_i})
\end{align}
% TODO algorithm for the optimization here
\section{Optimization step}\label{ch4:3}
In algorithm \ref{algorithm:lkjtm_obj}, first initialize the model and variational parameters. Then, for each epochs, we obtain the transformer embedding $ \rho $ from transformer. After that, the topic embedding $ \beta $ is computed by taking softmax of dot-product of $ \rho $ and $ \alpha $. Then a minibatch $ \mathcal{B} $ is selected for 
\\
\begin{algorithm}[H]
Initialize model and variational parameters\\
\For{epoch $i=1,2,\dots N$}{
Compute the trnasformer embedding $ \rho $\\
Compute $ \beta=\text{softmax}(\rho^\top\alpha) $\\
Choose a minibatch $ \mathcal{B} $ of documents\\
\ForEach{document d in $ \mathcal{B} $}{
Compute $ \mu_d=\text{NN}(x_d;\nu_\mu) $\\
Compute $ \sigma_d=\text{NN}(x_d;\mu_\sigma) $\\
Sample $ L\sim \text{LKJChol}(\gamma) $ \\
Sample $\theta_d\sim\mathcal{LN}(\mu,\sigma LL^\top\sigma)$\\
\ForEach{word position n in docuemnt $ N_d $}{
Sample word $ w_{dn}\sim \sigma(\beta_{w_{dn}}\theta_{d}) $
}
}
Estimate ELBO loss $ \text{L}_\text{ELBO}$ from Eq. \ref{eq:elbo_8}\\
Compute Transformer loss $ \text{L}_\text{CrossEntropy}$ from Eq. \ref{eq:crossentropy}\\
Compute the total loss $ \text{L}=\text{L}_\text{ELBO}+\text{L}_\text{CrossEntropy} $\\
Compute the stochastic gradient via backpropagation\\
Take a stochastic gradient step\\
Update model parameters ($\rho,\alpha,$)\\
Update variational parameters ($ \ $)
}
\label{algorithm:lkjtm_obj}
\caption{Topic modeling with the LKJTM}
\end{algorithm}
%\begin{algorithm}[H]
%Initial $ \theta^{(0)} $ randomly\\
%\While{Not Converge}{
%Sample a document d uniformly from dataset $ \mathcal{D} $\\
%For all k, initial $ \gamma^{d}_{k}=1 $\\
%\While{Not Converge}{
%\For{$ i=1,\cdots,N_d $}{
%\begin{align*}
%\phi_{ik}^{d}\propto\exp{\mathbb{E}}[\log\pi^d_k]+\mathbb{E}[\log\beta_{k,w_i^d}]
%\end{align*}
%}
%Set $ \gamma^{d}=\alpha+\sum_{i=1}^{N_d}\phi_i^d $
%}
%Take a stochastic gradient step $ \theta^{t}=\theta^{t-1}+\epsilon_t+\triangledown_\theta\mathcal{L}_d $
%}
%\caption{SVI for LDA}
%\end{algorithm}