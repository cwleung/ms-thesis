LDA is a prevelent 
In this chapter, we give a detailed explanation and procedure of LKJTM to be implemented.
% # CTM
\section{Related works}
Correlated Topic Model (CTM)\cite{blei_correlated_2007} is the original work that proposed to alleviate the problem LDA, which did not utilize the topic information between correlated topics. The proposed model replaced Dirichlet distribution with a logistic-normal prior with covariance matrix to represent the relationship between topics.
%% Embedded Topic Models
There have been a several of works focus on word embedding and topic model. Major of them combined statistical model and embedding approach to model topic distribution. In other words, representing a word by mapping every single word into continuous space.
% # CGTM
Xun \cite{xun_correlated_2017} employed words embedding into Correlated Topic Model, the new correlated topic model as Correlated Gaussian Topic Model (CGTM). In their paper they make use of word embedding space and model the correlation between topics by calculation of similarity between words in the embedding space.
% # CTMTE
Similarly, He\cite{he_efficient_2017} proposed Correlated Topic Modeling with Topic Embedding (CTMTE), which transformed the topic distribution previously obtained into lower dimension topic embedding space. The correlation between topics were directly computed through the similarity calculation in the vector space. The paper stated it reduces the running time as a scalable framework into large applications.
\section{Model description}
The TETM utilizes the Transformer as embedding to the topic-word representations. To compare with the original topic model, the word-topic distribution $ \beta $ is the 
First, the topic embedding embeds the vocabulary into L-dimensional space, which is by Transformer embedding. Second, the context embedding maps the embedding into K-dimensional space . 
In the generative process, the LKJTM uses the topic embedding to form a per-topic vector to represent the meaning over the vocabulary. 
% Generative process
The generative process of the $ d^{th} $ document is the following:\\
\begin{algorithm}[H]
\For{document d in D}{
Sample topic distribution $ \theta_d\sim \mathcal{LN}(0,I) $\\
\For{word position n in $ N_d $}{
Sample word $ w_{d,n}\sim \sigma(\theta_d(\rho^\top\alpha)_{\cdot,w_{d,n}}) $
}
}
\caption{Generative Process for TETM}
\label{algorithm:lkjtm}
\end{algorithm}
From algorithm \ref{algorithm:lkjtm}, starting from step 1, the topic proportion $ \theta_d $ is drawn from the logistic-normal distribution $ \mathcal{LN}(\cdot) $ with zero mean and identical covariance.
From Step 2-a, for each word position $ n $ in document $ d $, a topic assignment to word $ w_{dn} $ is drawn from categorical distribution $ Cat(\theta_d) $ parameterized by topic proportion $ \theta_d $
Step 2-b, the model draw a word from embedding of the vocabulary $ \rho $ and the assigned topic embedding $ \alpha_{z_{dn}} $ to draw the observed word from the assigned topic, as given by $ z_{dn} $. The embedding is applied softmax function to make them topic distribution.
The TETM likelihood uses a matrix of word embedding $ \rho $, a representation of the vocabulary in a lower dimensional space. In practice, it can either rely on previously fitted embeddings as part of the fitting procedure, it simultaneously finds topics and an embedding space.

% TODO LKJ Correlation Distribution Formulation

% TODO Transformer Embeddings
% Transformer embeddings
\subsection{Transformer Embeddings}
Following the ETM architecture, we modify topic-word distribution as an embedding and put transformer embedding to work into it. 
\begin{equation}\label{eq:transformer_embedding}
\beta\sim\text{softmax}(\rho^\top\alpha)
\end{equation}
Equation \ref{eq:transformer_embedding}, 
%% Joint Distribution
\subsection{Joint Distribution}
We give a description of the joint distribution, $ W,Z,\theta $ and $ \Sigma $ are variables and $ \beta, \mu $ and $ \gamma $ are latent variables. $ W $ is the word likelihood from the document collections, $ Z $ represents the topic-word assignment, $ \theta $ models the topic distribution for each document, and $ \Sigma $ is the covariance matrix which depends on the document-topic distribution $ \theta $.
\begin{align*}
p(w,\theta,\alpha)=\prod_{d=1}^{D}\int p(\theta_d)\prod_{i=1}^{N_d}p(w|\theta_d,\alpha)d\theta_d
\end{align*}
by taking  log on the joint probability, we obtain a objective function for optimization
% Marginal Likelihood
\subsection{Marginal Likelihood}
The parameter $ \alpha $ is the topic embedding on the word embedding space of dimension K. We take a log marginal likelihood of the document,
\begin{align*}
\mathcal{L}(\alpha)=\sum_{d=1}^{D}\log p(w_d|\alpha)
\end{align*}
\begin{align*}
p(w_d|\alpha)=\int p(\theta_d|\mu,\Sigma)\prod_{n=1}^{N_d}p(w_{dn}|\theta_d,\alpha)d\theta_d
\end{align*}
the conditional distribution $ p(w_{dn}|\theta_d,\alpha) $ marginalize out the the topic assignment $ z_{dn} $,
\begin{align*}
p(w_{dn}|\theta_d,\alpha)=\sum_{k=1}^{K}\theta_{dk}\beta_{k,w_{dn}}
\end{align*}
where $ \beta $ represent the topic-word distribution, composed of transformer embedding $ \rho $ and topic embedding $ \alpha $, in such case
\begin{align*}
\beta_{kv}=\text{softmax}(\rho^\top\alpha)_{v}.
\end{align*}
then we take the amortized inference to take out neural network for the representation,
\begin{align*}
p(w_{d}|N(\mu_\theta(w),\sigma_\theta(w)),\beta)=\mathbb{E}_{\theta\sim N(\mu_\theta(w),\sigma_\theta(w))}\left[w^\top\sigma(\beta\theta)\right]
\end{align*}
Reparametrization Trick $ \theta=\mu+\sigma^{1/2}\epsilon $\\
\begin{align*}
p(w_{d}|\mu_\theta(w)+\epsilon\sigma^{1/2}_\theta(w),\beta)=\mathbb{E}_{\epsilon\sim N(0,1)}\left[w_d^\top\sigma(\beta(\mu_\theta(w)+\epsilon\sigma^{1/2}_\theta(w)))\right]
\end{align*}